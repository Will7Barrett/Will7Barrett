\documentclass[a4paper,12pt]{article} 

\usepackage[top = 2.5cm, bottom = 2.5cm, left = 2.5cm, right = 2.5cm]{geometry} 

% packages
\usepackage{amsmath, amsfonts, amsthm} % basic math packages
\usepackage{tikz} % for making illustrations
\usetikzlibrary{shapes.arrows, arrows, decorations.markings, positioning}
\usetikzlibrary{calc}
\usetikzlibrary{3d}
\usepackage{graphicx} % for importing images
\usepackage{xcolor} % more color options
\usepackage{colortbl}
\usepackage{multicol} % for making two-column lists
\usepackage{hyperref} % for hyperlinking
%\hypersetup{colorlinks=true, urlcolor=cyan,}
\usepackage{mathabx}
\usepackage{cleveref}
\usepackage{subfig}
\usepackage{array}
\usepackage{wrapfig}
\usepackage{bbm}
\usepackage{fancyhdr}
\usepackage{algorithm, algorithmicx, algpseudocode}
\usepackage{stmaryrd}
\usepackage{physics}


% The following two packages - multirow and booktabs - are needed to create nice looking tables.
\usepackage{multirow} % Multirow is for tables with multiple rows within one cell.
\usepackage{booktabs} % For even nicer tables.

% As we usually want to include some plots (.p\partial f files) we need a package for that.
\usepackage{graphicx} 

% The default setting of LaTeX is to indent new paragraphs. This is useful for articles. But not really nice for homework problem sets. The following command sets the indent to 0.
\usepackage{setspace}
\setlength{\parindent}{0in}

% Package to place figures where you want them.
\usepackage{float}

% The fancyhdr package let's us create nice headers.
\usepackage{fancyhdr}

% theorems, lemmas, examples, etc.
\newtheorem{theorem}{Theorem}[section]
% \newtheorem{corollary}{Corollary}[theorem]
% \newtheorem{lemma}[theorem]{Lemma}
\newtheorem{example}[theorem]{Example}
\newtheorem{lemma}[theorem]{Lemma}
\theoremstyle{definition}
\newtheorem{definition}{Definition}[section]
\theoremstyle{remark}
\newtheorem*{remark}{Remark}
\newtheorem*{solution}{Solution}

\def\mydefb#1{\expandafter\def\csname bf#1\endcsname{\mathbf{#1}}}
\def\mydefallb#1{\ifx#1\mydefallb\else\mydefb#1\expandafter\mydefallb\fi}
\mydefallb aAbBcCdDeEfFgGhHiIjJkKlLmMnNoOpPqQrRsStTuUvVwWxXyYzZ\mydefallb

\def\mydefb#1{\expandafter\def\csname cal#1\endcsname{\mathcal{#1}}}
\def\mydefallb#1{\ifx#1\mydefallb\else\mydefb#1\expandafter\mydefallb\fi}
\mydefallb aAbBcCdDeEfFgGhHiIjJkKlLmMnNoOpPqQrRsStTuUvVwWxXyYzZ\mydefallb

%% Change this to just the normal N,Z,R,C,P,E
\def\mydefb#1{\expandafter\def\csname bb#1\endcsname{\mathbb{#1}}}
\def\mydefallb#1{\ifx#1\mydefallb\else\mydefb#1\expandafter\mydefallb\fi}
\mydefallb CEGIKNPQRST\mydefallb

\newcommand{\half}{\frac{1}{2}}
\newcommand{\Curl}[]{\text{curl}}
\DeclareMathOperator{\sgn}{sgn}
\DeclareMathOperator*{\argmax}{arg\,max}
\DeclareMathOperator*{\argmin}{arg\,min}
\newcommand{\matlab}{\textsc{Matlab}}


%%%%%%%%%%%%%%%%%%%%%%%%%%%%%%%%%%%%%%%%%%%%%%%%
% 3. Header (and Footer)
%%%%%%%%%%%%%%%%%%%%%%%%%%%%%%%%%%%%%%%%%%%%%%%%

% To make our document nice we want a header and number the pages in the footer.

\pagestyle{fancy} % With this command we can customize the header style.

\fancyhf{} % This makes sure we do not have other information in our header or footer.

\lhead{\footnotesize MATH 276:  Homework  \# 7}% \lhead puts text in the top left corner. \footnotesize sets our font to a smaller size.

%\rhead works just like \lhead (you can also use \chead)
\rhead{\footnotesize Barrett (wkbarre)} %<---- Fill in your lastnames.

% Similar commands work for the footer (\lfoot, \cfoot and \rfoot).
% We want to put our page number in the center.
\cfoot{\footnotesize \thepage} 

\newcommand{\N}{\mathbb{N}}
\newcommand{\Z}{\mathbb{Z}}
\newcommand{\Q}{\mathbb{Q}}
\newcommand{\R}{\mathbb{R}}
\newcommand{\C}{\mathbb{C}}
\newcommand{\F}{\mathbb{F}}
\newcommand{\parx}{\frac{\partial }{\partial x}}
\newcommand{\parxy}{\frac{\partial^2 }{\partial x \partial y}}
\newcommand{\pary}{\frac{\partial}{\partial y}}
\newcommand{\parz}{\frac{\partial}{\partial z}}


\begin{document}
	\thispagestyle{empty} % This command disables the header on the first page. 
	
	\begin{tabular}{p{15.5cm}} % This is a simple tabular environment to align your text nicely 
		{\large \sc Pre-Math 385 Practice} \\
		Emory University \\ Summer 2025 \\ ERSC \\
		\hline % \hline produces horizontal lines.
		\\
	\end{tabular} % Our tabular environment ends here.
	
	\vspace*{0.3cm} % Now we want to add some vertical space in between the line and our title.
	
	\begin{center} % Everything within the center environment is centered.
		{\Large \bf Discrete and Combinatorial Mathematics, An Applied Introduction, 3rd Edition} % <---- Don't forget to put in the right number
		\vspace{2mm}
		
		% YOUR NAMES GO HERE
		{\bf William Barrett}\\ % <---- Fill in your names here!
		
	\end{center}  
	
	\vspace{0.4cm}
%Fundamentals of Discrete Mathematics
\section{Enumeration [Independent]}
\subsection{Exercises 1.1 and 1.2}
\framebox{\begin{minipage}{0.98\textwidth}
  \textbf{DEFINITIONS:}
  \begin{enumerate}
\item\textbf{The Rule of Sum:} If a task can be performed in $m$ ways, while a second task can be performed in $n$ ways, and the two tasks cannot be performed simultaneously, then performing either task can be accomplished in any one of $m+n$ ways.
\item\textbf{The Rule of Product:} If task can be performed into first and second stages, and if there are $m$ possible outcomes for the first stage and if, for each of these outcomes, there are $n$ possible outcomes for the second stage, then the total procedure can be carried out, in designated order, in $mn$ ways
\item\textbf{Permutations:} In general, if there are $n$ distinct objects, denoted $a_1,...,a_n$, and $r\in\{1,..,n\}$, by the rule of product, the number of permutations of size $r$ for the $n$ objects is \begin{equation*}
    (n+1-1)\times(n+1-2)\times\dots\times(n+1-r)=\frac{n!}{(n-r)!}
\end{equation*}
\item\textbf{Permutations:} If there are $n$ objects with $n_1$ as a first type, $n_2$ as a second, $\dots$, and $n_r$ of an $r$th type, where $\sum_{i=1}^rn_i=n$, then there are $\frac{n!}{\prod^r_{i=1}(n_i!)}$ (linear) arrangements of the given $n$ arguments. (Objects of the same type are indistinguishable.)
    \end{enumerate}

\end{minipage}}

\begin{enumerate}
    \item \textit{Republicans and Democrats}
    \begin{enumerate}
        \item If there are five Democratic candidates and eight Republican candidates, the law of sums tells us there are $5+8=13$ possible candidates who can win the election, as the processes of a Democrat winning and a Republican winning cannot occur simultaneously. %Rule of Sum 1
        \item If there are five Democratic candidates and eight Republican candidates, the law of products tells us that there are $5\times 8=40$ possible pairs of candidates who will win the primary, as there are $5$ and $8$ outcomes for each respective stage. %Rule of Products 1
        \item  Part (a) utilizes the law of sums, while part (b) utilizes the law of products.
    \end{enumerate}
\item We know that the number of possible license plates with only vowel letters and only even numbers (allowing for repetition) is equal to $5\times5\times5\times5\times5\times5=5^6=15625$. This is because there are five even digits and five vowels. %Rule of Products 2
\item \textit{Buick Automobiles}
\begin{enumerate}
    \item Invoking the law of products for four simultaneous processes, we find the number of unique Buick automobiles to be \begin{equation*}
        4\times12\times3\times2=288
    \end{equation*} %Rule of Products 3
    \item Invoking the law of products for four simultaneous processes (while fixing the outcome of the second), we find the number of unique blue Buick automobiles to be\begin{equation*}
        4\times1\times3\times2=24
    \end{equation*}
\end{enumerate}
\item \textit{Big Pharma}
\begin{enumerate}
    \item There are four simultaneous processes with a selection pool of ten for all of the processes, not allowing for repetition. Invoking the law of products, we find the number of permutations to be \begin{equation*} 
        \frac{10!}{(10-4)!}=5040
    \end{equation*} %permutations 1
    \item \textit{What About the Physicians?}
    \begin{enumerate}
        \item There are four simultaneous processes without repetition with a selection pool of nine for three of the processes, and a selection pool of three for one of the processes, not allowing for repetition. Invoking the law of products, we find the number of permutations to be \begin{equation*} 
        3\times\frac{ 9!}{(9-3)!}=1512
    \end{equation*}%permutations 2
    Which also happens to be equal to the probability of a physician being president, $0.3$, times the total number of combinations given in part (a).
    \item There are four simultaneous processes with a selection pool of three for one of the processes, and a selection pool of seven for the other three processes, not allowing for repetition. Invoking the law of products, we find the number of permutations to be: \begin{equation*} 
        3\times7\times6\times5=3\times\frac{ 7!}{(7-3)!}
    \end{equation*}%permutations 3
    \item There are four simultaneous processes with a selection pool of three for one of the processes, and a selection pool of nine for the other three processes, not allowing for repetition. Invoking the law of products, we find the number of permutations to be: \begin{equation*}
        3\times\frac{9!}{(9-3)!}=1512
    \end{equation*} Notice this result is the same number of combinations of slates with a physician for president. This is because, mathematically, having at least one physician on the board is the same as having that role be fixed as president. 
\end{enumerate}
\end{enumerate}
\item There are six simultaneous processes with repetition, each with two, two, one, ten, ten, and two possibilities, respectively.  We find the number of combinations to be: \begin{equation*}
    2\times2\times1\times10\times10\times2=800
\end{equation*} There are eight hundred possible license plates matching this description.
\item \textit{Fundraising}
\begin{enumerate}
    \item Given thirty distinct people, for a permutation of size 8 we find\begin{equation*}
        \frac{30!}{(30-8)!}=2.35989936\times10^{11}
    \end{equation*} Or over two hundred billion combinations. 
    \item Given that Roberta and Candice are in the top three, there are $3!=6$ combinations where they are in the top three. The second event is the other twenty seven people competing for the latter five spots (a 27 object permutation of size 5), which we find to be \begin{equation*}
        \frac{27!}{(27-5)!}=9,687,600
    \end{equation*} By the law of products, there are\begin{equation*}
        3!\frac{27!}{22!}=58,125,600
    \end{equation*} combinations.
\end{enumerate}
\item Given nine toppings with no limit on the number of toppings one can get, assuming someone doesn't repeat toppings there are $2^9=512$ combinations.
\item \textit{Batch Processing}
\begin{enumerate}
    \item With no restrictions there are $12!=479,001,600$ combinations.
    \item If there are four priority programs and eight nonpriority programs, the combinations can be represented as the product of permutations:\begin{equation*}
        4!8!=967,680
    \end{equation*}
    \item If there are four priority programs, five of lesser priority, and three of least priority, the combinations can be represented as the product of three permutations: \begin{equation*}
        4!5!3!=17,280
    \end{equation*}
\end{enumerate}
\item \textit{Patter's Pastry Parlor}
\begin{enumerate}
    \item If there are eight different types of pastry and six different types of muffins, then by the law of sums there are $8+6=14$ different bakery items. In addition to bakery items, there are five types of coffee, six types of tea, hot cocoa, and orange juice. By law of sums there are $5+6+1+1=13$ beverages. Considering this person is buying a medium sized drink, we fix drink size. By the law of products, there are $14\times13=182$ ways to order one bakery item and one medium drink. 
    \item If there are fourteen bakery items and six types of muffins, by the law of products, there are $14\times6=84$ combinations of edible items Carol can order. Given that there are five types of coffee, six types of tea, and three possible sizes, by the law of products, there are $5\times6\times3=90$ combinations of beverages Carol can order. By the law of products, there are $84\times90=7,560$ combinations Carol can get.
    \item If there are eight types of pastry, six types of muffin, and fourteen total bakery items, by the law of products there are $8\times6\times14=672$ different food combinations Carol can order. Given that there are six types of tea, one type of orange juice, five types of coffee, and three sizes of beverage, by the law of products there are $6\times 1\times 5\times 3=90$ combinations of beverages Carol can order. By the law of products, there are $672\times90=60,480$ combinations Carol can get for her coworkers.
\end{enumerate}
\item Given that Pamela must have one book on each shelf, we find the number of combinations to be a fifteen object permutation of size two. \begin{equation*}
    \frac{15!}{(15-2)!}=210
\end{equation*} For the other thirteen books, they can be arranged in any which way, making the number of combinations $13!$ By the law of products, the total number of arrangements is \begin{equation*}
    \frac{15!}{13!}13!=15!=1.307674368\times10^{12}
\end{equation*}
\end{enumerate}

\subsection{Exercises 1.3}
\subsection{Exercises 1.4}
\section{Logic [Independent]}
\section{Set Theory [Dependent on 1,2]}
\section{Properties of the Integers: Mathematical Induction [Dependent on 1,2,3]}
\section{Relations and Functions}
\section{Languages: Finite State Machines}
\section{Relations: The Second Time Around}
%Further Topics in Enumeration
\section{The Principle of Inclusion and Exclusion}
\section{Generating Functions}
\section{Recurrence Relations}
%Graph Theory and Applications
\section{An Introduction to Graph Theory}
\section{Trees}
\section{Optimization and Matching}
%Modern Applied Algebra
\section{Rings and Modular Arithmetic}
\section{Boolean Algebra and Switching Functions}
\section{Groups, Coding Theory, and Polya's Method of Enumeration}
\section{Finite Fields and Combinatorial Designs}
\end{document}